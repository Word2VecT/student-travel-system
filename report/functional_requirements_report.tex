\documentclass[a4paper, 12pt]{ctexart}
\usepackage{geometry}
\usepackage{hyperref}
\usepackage{enumitem}
\geometry{left=2cm, right=2cm, top=2.5cm, bottom=2.5cm}

\title{学生游学系统功能需求报告}
\author{队长:唐梓楠\ 2022211404 \\ 成员:马伟昊\ 2022211401 \\\phantom{成员:}张\quad 付\ 2022211396}
\date{2024年2月——2024年6月}

\begin{document}

\maketitle

\section{引言}
学生经常利用假期去各地游学。学生游学系统可以帮助学生管理自己的游学活动,具备游学推荐、游学路线规划、场所查询、游学日记管理等功能。

\section{数据要求}
\begin{itemize}
    \item 景区和校园数量至少200个,景区和校园内部可以一致;
    \item 景区和校园内建筑物(景点、教学楼、办公楼、宿舍楼)数不少于20个;其它服务设施不少于10种(商店、饭店、洗手间、图书馆、食堂、超市、咖啡馆等),数量不少于50个;
    \item 建立景区和校园内部道路图,包括各种建筑物、服务设施等信息,边数不能少于200条(尽量接近真实景区和校园);
    \item 系统用户数不少于10人;
\end{itemize}

\section{功能需求}
\subsection{游学推荐}
\begin{itemize}
    \item 学生可以根据自己的喜好选择不同的景点和学校作为游学目的地。
    \item 在游览前,系统会向学生推荐游学景点和学校,可以按照游学热度、评价和个人兴趣进行推荐。
    \item 推荐算法基础要求为排序算法,可以按照用户选择的热度和评价进行排序。
    \item 学生可以输入景点和学校的名称、类别、关键字等进行查询,查询结果有多项时,可以对查询结果按照热度和评价进行排序。
\end{itemize}

\subsection{游学路线规划}
\begin{itemize}
    \item 当进入景区或者学校后,学生可以输入目标景点或者场所信息,系统会为学生规划从当前位置出发到达景点或者场所的最优游学线路(核心算法为最短路径算法)。
    \item 学生可以输入多个目标景点或者场所信息,系统会为学生规划从当前位置出发,参观多个景点或者场所的最优游学线路(核心算法为途经多点最短路径算法)。
    \item 路线规划策略包括:最短距离策略、最短时间策略、交通工具的最短时间策略(选做)。
\end{itemize}

\subsection{场所查询}
\begin{itemize}
    \item 在景区或者学校内部时,选中某个景点或者场所,会找出附近一定范围内的超市、卫生间等设施,并根据距离进行排序。
    \item 可以通过选择类别对结果进行过滤。
    \item 可以由用户输入类别名称查找某个地点附近的服务设施,并根据距离进行排序。
\end{itemize}

\subsection{游学日记管理}
\begin{itemize}
    \item 学生游学过程中或者游学结束时可以撰写游学日记,通过文字的方式记录游学内容。
    \item 需要对所有学生的游学日记进行统一的管理。
    \item 学生可以浏览和查询所有学生的游学日记,游学日记的浏览量即为该日记的热度,每位同学浏览完可以对游学日记进行评分。
    \item 学生在浏览所有游学日记时,可以按照日记热度、评价和个人兴趣进行推荐,推荐算法基础要求为排序算法。
    \item 学生可以输入游学目的地,对目的地相关的游学日记根据热度和评分进行排序。
    \item 学生可以输入游学日记的名称进行精确查询。
    \item 可以按日记内容进行全文检索(核心算法为文本搜索)。
    \item 可以对游学日记进行压缩存储(核心算法为无损压缩)。
\end{itemize}

\section{选做功能需求}
\begin{itemize}
    \item 设计导航功能的图形界面,包括地图展示和输出路径展示;
    \item 室内导航策略:模拟教学楼的结构和景区内博物馆等建筑物的内部结构,进行室内导航,包括大门到电梯的导航、楼层间的电梯导航和楼层内到房间的导航;
    \item 交通工具的最短时间策略:校区内可以选择自行车和步行,选择自行车时,只能走自行车道路,默认自行车在校区任何地点都有;景区内可以选择步行和电瓶车,选择电瓶车时只能走电瓶车路线,电瓶车路线固定,默认上车即走;不同交通工具可以选择时,考虑不同拥挤度的情况下时间最短(时间最短的线路,可以是多种交通工具混合);
    \item 美食推荐:在选中游览景点和学校后,可以按照用户选择的热度、评价和距离进行排序,并根据菜系进行过滤;
    \item 采用推荐算法(基于内容推荐、协同过滤推荐等)进行景点、学校、美食和游学日记的推荐;
    \item 使用AIGC算法根据拍摄的景点或者学校的照片进行游学动画生成。
\end{itemize}

\section{总结}
本系统旨在通过设计和实现学生游学系统,帮助学生更好地管理游学活动,提高游学体验。系统功能涵盖游学推荐、游学路线规划、场所查询和游学日记管理,并提供多种选做功能以增强系统的实用性和智能性。

\end{document}
