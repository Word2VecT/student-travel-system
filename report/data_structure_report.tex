\documentclass{ctexart}
\usepackage{geometry}
\usepackage{graphicx}
\usepackage{hyperref}
\usepackage{booktabs}
\usepackage{minted}
\usepackage{tikz}
\usetikzlibrary{shapes.geometric, arrows}

\tikzstyle{block} = [rectangle, rounded corners, minimum width=3cm, minimum height=1cm, text centered, draw=black, fill=blue!30]
\tikzstyle{line} = [draw, -latex']

\newminted{sql}{breakanywhere,breaklines,linenos,frame=single}

\title{学生游学系统 \\ 数据结构说明报告}
\author{队长:唐梓楠\ 2022211404 \\ 成员:马伟昊\ 2022211401 \\ 张付\ 2022211396}
\date{\today}

\hypersetup{colorlinks=true,linkcolor=black,filecolor=magenta,urlcolor=cyan}

\begin{document}

\maketitle

\tableofcontents

\newpage

\section{引言}
本报告详细描述了学生游学系统的数据结构设计,包括用户表、景点表和游记表的字段及其含义。通过对各表的设计说明,可以清晰地了解系统的数据处理逻辑和实现方法。

\section{数据结构}

\subsection{用户表}
用户表用于存储用户的基本信息,包括用户ID、用户名和密码。每个用户在系统中都有唯一的ID标识。

\begin{center}
    \begin{tabular}{ccc}
        \toprule
        字段名      & 类型      & 描述         \\
        \midrule
        id       & INT     & 用户ID,主键,自增 \\
        username & VARCHAR & 用户名,唯一,不为空 \\
        password & VARCHAR & 密码,不为空     \\
        \bottomrule
    \end{tabular}
\end{center}

\paragraph{字段说明:}
\begin{itemize}
    \item \textbf{id}:用户的唯一标识符,自动递增。
    \item \textbf{username}:用户的登录名,必须唯一且不为空。
    \item \textbf{password}:用户的密码,存储为加密后的字符串。
\end{itemize}

\subsection{景点表}
景点表用于存储景点的基本信息,包括景点ID、名称、热度、类别、评分、价格、地址、地区和描述。每个景点在系统中都有唯一的ID标识。

\begin{center}
    \begin{tabular}{ccc}
        \toprule
        字段名         & 类型      & 描述         \\
        \midrule
        id          & INT     & 景点ID,主键,自增 \\
        name        & VARCHAR & 景点名称,不为空   \\
        popularity  & INT     & 热度,不为空     \\
        category    & VARCHAR & 类别,可以为空    \\
        rating      & FLOAT   & 评分,不为空     \\
        price       & FLOAT   & 价格,不为空     \\
        address     & VARCHAR & 地址,不为空     \\
        region      & VARCHAR & 地区,不为空     \\
        description & TEXT    & 景点描述,不为空   \\
        \bottomrule
    \end{tabular}
\end{center}

\paragraph{字段说明:}
\begin{itemize}
    \item \textbf{id}:景点的唯一标识符,自动递增。
    \item \textbf{name}:景点的名称,不为空。
    \item \textbf{popularity}:景点的热度,用于排序和推荐。
    \item \textbf{category}:景点的类别,例如5A、4A等,可以为空。
    \item \textbf{rating}:景点的评分,范围为0-5。
    \item \textbf{price}:景点的价格,门票费用。
    \item \textbf{address}:景点的详细地址。
    \item \textbf{region}:景点所属的地区,例如某个城市或区域。
    \item \textbf{description}:景点的详细描述。
\end{itemize}

\subsection{游记表}
游记表用于存储游记的基本信息,包括游记ID、景点ID、内容、浏览量、评分、用户名、标题和评分次数。每篇游记在系统中都有唯一的ID标识。

\begin{center}
    \begin{tabular}{ccc}
        \toprule
        字段名             & 类型      & 描述          \\
        \midrule
        id              & INT     & 游记ID,主键,自增  \\
        destination\_id & INT     & 景点ID,外键,不为空 \\
        content         & TEXT    & 内容,不为空      \\
        views           & INT     & 浏览量,默认值为0   \\
        rating          & FLOAT   & 评分,默认值为0    \\
        username        & VARCHAR & 用户名,不为空     \\
        title           & VARCHAR & 标题,不为空      \\
        rate\_cnt       & INT     & 评分次数,默认值为0  \\
        \bottomrule
    \end{tabular}
\end{center}

\paragraph{字段说明:}
\begin{itemize}
    \item \textbf{id}:游记的唯一标识符,自动递增。
    \item \textbf{destination\_id}:游记所属的景点ID,外键关联景点表。
    \item \textbf{content}:游记的详细内容,可以包含文字和图片。
    \item \textbf{views}:游记的浏览量,用于计算热度。
    \item \textbf{rating}:游记的评分,范围为0-5。
    \item \textbf{username}:发布游记的用户的用户名。
    \item \textbf{title}:游记的标题。
    \item \textbf{rate\_cnt}:游记的评分次数,用于计算平均评分。
\end{itemize}

\section{数据库设计}
系统采用MySQL数据库,数据库中的每张表分别对应上述的数据结构设计。

\subsection{用户表设计}
\begin{sqlcode}
    CREATE TABLE `user` (
    `id` int unsigned NOT NULL AUTO_INCREMENT,
    `username` varchar(120) CHARACTER SET utf8mb4 COLLATE utf8mb4_0900_ai_ci NOT NULL,
    `password` varchar(120) CHARACTER SET utf8mb4 COLLATE utf8mb4_0900_ai_ci NOT NULL,
    PRIMARY KEY (`id`),
    UNIQUE KEY `UC_username` (`username`)
    ) ENGINE=InnoDB AUTO_INCREMENT=13 DEFAULT CHARSET=utf8mb4 COLLATE=utf8mb4_0900_ai_ci;
\end{sqlcode}

\subsection{景点表设计}
\begin{sqlcode}
    CREATE TABLE `destination` (
    `id` int unsigned NOT NULL AUTO_INCREMENT,
    `name` varchar(120) CHARACTER SET utf8mb4 COLLATE utf8mb4_0900_ai_ci NOT NULL,
    `popularity` int NOT NULL,
    `category` varchar(10) CHARACTER SET utf8mb4 COLLATE utf8mb4_0900_ai_ci DEFAULT NULL,
    `rating` float NOT NULL,
    `price` float NOT NULL,
    `address` varchar(255) CHARACTER SET utf8mb4 COLLATE utf8mb4_0900_ai_ci NOT NULL,
    `region` varchar(10) NOT NULL,
    `description` text CHARACTER SET utf8mb4 COLLATE utf8mb4_0900_ai_ci NOT NULL,
    PRIMARY KEY (`id`)
    ) ENGINE=InnoDB AUTO_INCREMENT=3991 DEFAULT CHARSET=utf8mb4 COLLATE=utf8mb4_0900_ai_ci;
\end{sqlcode}

\subsection{游记表设计}
\begin{sqlcode}
    CREATE TABLE `travel_note` (
    `id` int unsigned NOT NULL AUTO_INCREMENT,
    `destination_id` int unsigned NOT NULL,
    `content` text CHARACTER SET utf8mb4 COLLATE utf8mb4_0900_ai_ci NOT NULL,
    `views` int NOT NULL DEFAULT '0',
    `rating` float NOT NULL DEFAULT '0',
    `username` varchar(120) CHARACTER SET utf8mb4 COLLATE utf8mb4_0900_ai_ci NOT NULL,
    `title` varchar(120) CHARACTER SET utf8mb4 COLLATE utf8mb4_0900_ai_ci NOT NULL,
    `rate_cnt` int NOT NULL DEFAULT '0',
    PRIMARY KEY (`id`),
    KEY `fk_destination` (`destination_id`),
    CONSTRAINT `fk_destination` FOREIGN KEY (`destination_id`) REFERENCES `destination` (`id`)
    ) ENGINE=InnoDB AUTO_INCREMENT=25 DEFAULT CHARSET=utf8mb4 COLLATE=utf8mb4_0900_ai_ci;
\end{sqlcode}

\section{数据关系图}
系统中各数据表之间的关系如下图所示:
\begin{center}
    \begin{tikzpicture}[node distance=2cm]

        % Nodes
        \node (user) [block] {用户表 (user)};
        \node (destination) [block, below of=user] {景点表 (destination)};
        \node (travel_note) [block, below of=destination] {游记表 (travel\_note)};

        % Lines
        \path [line] (user) -- (travel_note);
        \path [line] (destination) -- (travel_note);

    \end{tikzpicture}
\end{center}

\section{总结}
本报告详细描述了学生游学系统的数据结构设计。通过对用户表、景点表和游记表的详细说明,可以清晰地了解系统的数据处理逻辑和实现方法。系统采用MySQL数据库进行数据存储,各表的设计确保了数据的完整性和一致性。

\end{document}
