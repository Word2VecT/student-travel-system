\documentclass{ctexart}
\usepackage{geometry}
\usepackage{graphicx}
\usepackage{hyperref}
\usepackage{booktabs}

\title{学生游学系统 \\ 用户使用说明报告}
\author{队长:唐梓楠\ 2022211404 \\ 成员:马伟昊\ 2022211401 \\ 张付\ 2022211396}
\date{\today}

\hypersetup{colorlinks=true,linkcolor=black,filecolor=magenta,urlcolor=cyan}

\begin{document}

\maketitle

\tableofcontents

\newpage

\section{引言}
本报告旨在为用户提供学生游学系统的详细使用说明。通过本报告,用户可以了解系统的主要功能和使用方法,快速上手并充分利用系统的各项功能。

\section{系统概述}
学生游学系统是一个集用户管理、景点推荐、路线规划、场所查询和游记管理于一体的综合性平台。用户可以通过本系统进行游学规划、景点查询、撰写游记等活动。

\section{系统登录与注册}

\subsection{用户注册}
用户可以通过以下步骤进行注册:
\begin{enumerate}
    \item 打开系统首页,点击``需要注册?''按钮。
    \item 在注册页面填写用户名和密码,点击“提交”按钮。
    \item 系统提示注册成功,用户可以使用新注册的账户登录系统。
\end{enumerate}

\subsection{用户登录}
用户可以通过以下步骤进行登录:
\begin{enumerate}
    \item 打开系统首页。
    \item 在登录页面输入用户名和密码,点击“提交”按钮。
    \item 系统验证用户名和密码后,提示登录成功,进入系统主页面。
\end{enumerate}

\section{景点推荐}

\subsection{景点推荐}
系统会根据用户的偏好和历史记录推荐合适的景点。用户可以在推荐页面查看系统推荐的景点。

\subsection{搜索景点}
用户可以通过以下步骤搜索景点:
\begin{enumerate}
    \item 在推荐页面的搜索框中输入景点名称或关键词。
    \item 点击“搜索”按钮,系统会显示匹配的景点列表。
\end{enumerate}

\subsection{分类筛选}
用户可以通过以下步骤进行分类筛选:
\begin{enumerate}
    \item 在推荐页面选择景点类别、地区等筛选条件。
    \item 系统会根据筛选条件显示相应的景点列表。
\end{enumerate}

\subsection{偏好排序}
用户可以通过以下步骤进行偏好排序:
\begin{enumerate}
    \item 在推荐页面选择按照热度、价格等排序条件。
    \item 系统会根据排序条件对推荐景点列表进行排序。
\end{enumerate}

\section{游记管理}

\subsection{浏览游记}
用户可以通过以下步骤浏览其他用户的游记:
\begin{enumerate}
    \item 在游记管理页面查看游记列表,可以根据热度和评分排序。
    \item 点击某个游记的向下按钮,系统会显示该游记的详细内容。
\end{enumerate}

\subsection{撰写游记}
用户可以通过以下步骤撰写游记:
\begin{enumerate}
    \item 在游记管理页面点击“撰写游记”按钮。
    \item 在撰写游记页面输入游记标题和内容,可以添加图片。
    \item 点击“发布”按钮,系统会保存游记并显示发布成功提示。
\end{enumerate}

\subsection{评分游记}
用户可以通过以下步骤对游记进行评分:
\begin{enumerate}
    \item 在游记详情页面点击“评分”按钮。
    \item 选择评分等级,点击“提交”按钮,系统会记录评分并更新游记的平均评分。
\end{enumerate}

\subsection{推荐游记}
系统会根据游记的热度和评分推荐优秀的游记,用户可以在推荐页面查看推荐的游记。

\section{路线规划}

\subsection{单目标路线规划}
用户可以通过以下步骤进行单目标路线规划:
\begin{enumerate}
    \item 在路线规划页面选择一个目标景点。
    \item 系统会根据用户的当前位置规划最优路线,并显示路线图。
\end{enumerate}

\subsection{多目标路线规划}
用户可以通过以下步骤进行多目标路线规划:
\begin{enumerate}
    \item 在路线规划页面选择多个目标景点。
    \item 系统会规划覆盖所有目标景点的最优路线,并显示路线图。
\end{enumerate}

\subsection{选择出行方式}
用户可以通过以下步骤选择出行方式:
\begin{enumerate}
    \item 在路线规划页面选择步行或骑行方式。
    \item 系统会根据选择的出行方式规划相应的路线,并显示路线图。
\end{enumerate}

\section{场所查询}

\subsection{查询设施}
用户可以通过以下步骤查询景区或学校内部的设施:
\begin{enumerate}
    \item 在场所查询页面输入设施名称或选择设施类别。
    \item 点击“查询”按钮,系统会显示匹配的设施列表。
\end{enumerate}

\subsection{设施排序}
用户可以通过以下步骤对设施进行排序:
\begin{enumerate}
    \item 在设施查询结果页面选择排序条件,如距离或评分。
    \item 系统会根据选择的排序条件重新排列设施列表。
\end{enumerate}

\subsection{查看设施详情}
用户可以通过以下步骤查看设施详情:
\begin{enumerate}
    \item 在设施查询结果页面点击某个设施的名称。
    \item 系统会显示该设施的详细信息,包括位置、评分、用户评论等。
\end{enumerate}

\subsection{导航到设施}
用户可以通过以下步骤导航到选定的设施:
\begin{enumerate}
    \item 在设施详情页面点击“导航”按钮。
    \item 系统会根据用户的当前位置和选定的设施位置规划路线,并显示导航图。
\end{enumerate}

\section{常见问题}

\subsection{无法登录}
如果用户无法登录,可以检查以下事项:
\begin{itemize}
    \item 确认输入的用户名和密码是否正确。
    \item 检查网络连接是否正常。
    \item 如果仍然无法登录,可以联系系统管理员获取帮助。
\end{itemize}

\section{技术支持}
如果用户在使用系统过程中遇到问题,可以通过以下方式获取技术支持:
\begin{itemize}
    \item \textbf{在线帮助}:系统提供在线帮助文档,用户可以通过点击页面上的“帮助”按钮查看。
    \item \textbf{客服热线}:用户可以拨打客服热线获取技术支持。
    \item \textbf{电子邮件}:用户可以发送电子邮件至support@example.com获取帮助。
\end{itemize}

\section{总结}
本报告详细介绍了学生游学系统的使用方法,涵盖了系统的主要功能和操作步骤。通过本报告,用户可以快速上手并充分利用系统的各项功能,提升游学体验。

\end{document}
